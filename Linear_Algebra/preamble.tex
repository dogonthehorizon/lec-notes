\documentclass[12pt,twoside]{article}

%%%%%%% Pagestyle stuff %%%%%%%%%%%%%%%%%
 \usepackage{fancyhdr}                 %%
   \pagestyle{fancy}                   %%
   \fancyhf{} %delete the current section for header and footer
 \usepackage[paperheight=11in,%        %%
             paperwidth=8in,%        %%
             outer=1in,%             %%
             inner=1in,%             %%
             bottom=.7in,%             %%
             top=.7in,%                %%
             includeheadfoot]{geometry}%%
%    \columnsep 3em
   \setlength{\headwidth}{6in}       %% 
   \fancyhead[RO,LE]{\thepage}         %%
   \fancyhead[LO,RE]{\sectionname}     %%
   \setlength{\headheight}{14.5pt}     %%
   \raggedbottom                       %%
%%%%%%% End Pagestyle stuff %%%%%%%%%%%%%
 
 \usepackage{etex}          % For some reason, pdflatex breaks if I don't include the etex package
 \usepackage{amsmath}       % I think this gives me some symbols
 \usepackage{amsthm}        % Does theorem stuff
 \usepackage{amssymb}       % more symbols and fonts
 \usepackage{empheq}        % Some more extensible arrows, like \xmapsto
 \usepackage{enumitem}
 \usepackage{mathrsfs}      % Sheafy font \mathscr{}
% \usepackage{picinpar}     % for pictures in paragraphs
% \usepackage{tikz}
 \usepackage[nofancy]{rcsinfo}
 \usepackage[all]{xy}       % Include XY-pic
    \SelectTips{cm}{10}     % Use the nicer arrowheads
    \everyxy={<2.5em,0em>:} % Sets the scale I like
 \usepackage[colorlinks,
             linkcolor=black,
             pagebackref,
             bookmarksnumbered=true]{hyperref}
 \usepackage{hyperref}
 \usepackage{hevea}
 \loadcssfile{http://notes.binaryruminations.com/css/notes.css}

%%%%%%% Stuff for keeping track of sections %%%%%%%%%%%%%%%%%%%%%%%%%%%%
 \newcount\n
 \newcommand{\sektion}[2]{\stepcounter{section} \renewcommand{\thesection}{#1}\n=\count0 \newpage
\ifnum\count0=\n \ifnum\count0>1\ \newpage \fi\fi\section{#2} \gdef\sectionname{#1\quad #2, v.~\rcsInfoMonth-\rcsInfoDay}}
 \newcommand{\subsektion}[1]{\subsection*{#1} \addcontentsline{toc}{subsection}{#1}}
 \newcommand{\subsubsektion}[1]{\subsubsection*{#1}}
 %% This is the empty section title, before any section title is set %%
 \newcommand\sectionname{}                                           %%
%%%%%%% End stuff for keeping track of sections %%%%%%%%%%%%%%%%%%%%%%%

%%%%%%%%%%%%%%% Theorem Styles and Counters %%%%%%%%%%%%%%%%%%%%%%%%%%
 \renewcommand{\theequation}{\thesection.\arabic{equation}}         %%
 \makeatletter                                                      %%
    \@addtoreset{equation}{section} % Make the equation counter reset each section
    \@addtoreset{footnote}{section} % Make the footnote counter reset each section
                                                                    %%
 \newenvironment{warning}[1][]{%                                    %%
    \begin{trivlist} \item[] \noindent%                             %%
    \begingroup\hangindent=2pc\hangafter=-2                         %%
    \clubpenalty=10000%                                             %%
    \hbox to0pt{\hskip-\hangindent\manfntsymbol{127}\hfill}\ignorespaces%
    \refstepcounter{equation}\textbf{Warning~\theequation}%         %%
    \@ifnotempty{#1}{\the\thm@notefont \ (#1)}\textbf{.}            %%
    \let\p@@r=\par \def\p@r{\p@@r \hangindent=0pc} \let\par=\p@r}%  %%
    {\hspace*{\fill}$\lrcorner$\endgraf\endgroup\end{trivlist}}     %%
                                                                    %%
 \newenvironment{exercise}[1][]{\begin{trivlist}%                   %%
    \item{\bf Exercise\@ifnotempty{#1}{ #1}. }\it}{\end{trivlist}}   %%
 \newenvironment{solution}{\begin{trivlist}%                        %%
    \item{\it Solution.}}{\end{trivlist}}                           %%
                                                                    %%
 \def\newprooflikeenvironment#1#2#3#4{%                             %%
      \newenvironment{#1}[1][]{%                                    %%
          \refstepcounter{equation}                                 %%
          \begin{proof}[{\rm\csname#4\endcsname{#2~\theequation}%   %%
          \@ifnotempty{##1}{\the\thm@notefont \ (##1)}\csname#4\endcsname{.}}]%%
          \def\qedsymbol{#3}}%                                      %%
         {\end{proof}}}                                             %%
 \makeatother                                                       %%
                                                                    %%
 \newprooflikeenvironment{definition}{Definition}{$\diamond$}{textbf}%
 \newprooflikeenvironment{example}{Example}{$\diamond$}{textbf}     %%
 \newprooflikeenvironment{remark}{Remark}{$\diamond$}{textbf}       %%
                                                                    %%
 \theoremstyle{plain}                                               %%
 \newtheorem{theorem}[equation]{Theorem}                            %%
 \newtheorem*{claim}{Claim}                                         %%
 \newtheorem*{lemma*}{Lemma}                                        %%
 \newtheorem*{theorem*}{Theorem}                                    %%
 \newtheorem{lemma}[equation]{Lemma}                                %%
 \newtheorem{corollary}[equation]{Corollary}                        %%
 \newtheorem{proposition}[equation]{Proposition}                    %%
%%%%%%%%%%% End Theorem Styles and Counters %%%%%%%%%%%%%%%%%%%%%%%%%%

%%% These three lines load and resize a caligraphic font %%%%%%%%%
%%% which I use whenever I want lowercase \mathcal %%%%%%%%%%%%%%%
 \DeclareFontFamily{OT1}{pzc}{}                                 %%
 \DeclareFontShape{OT1}{pzc}{m}{it}{<-> s * [1.100] pzcmi7t}{}  %%
 \DeclareMathAlphabet{\mathpzc}{OT1}{pzc}{m}{it}                %%
                                                                %%
%%% and this is manfnt; used to produce the warning symbol %%%%%%%
 \DeclareFontFamily{U}{manual}{}                                %%
 \DeclareFontShape{U}{manual}{m}{n}{ <->  manfnt }{}            %%
 \newcommand{\manfntsymbol}[1]{%                                %%
    {\fontencoding{U}\fontfamily{manual}\selectfont\symbol{#1}}}%%
%%%%%%%%%%%%%%%%%%%%%%%%%%%%%%%%%%%%%%%%%%%%%%%%%%%%%%%%%%%%%%%%%%

%%%%%%%%%%%% Begin Macros %%%%%%%%%%%%%%%%%%%%%%%%%%%%%%%%%%%%%%%%%%%%%
 
 %Number Sets
 \newcommand{\CC}{\mathbb{C}}
 \newcommand{\NN}{\mathbb{N}}
 \newcommand{\QQ}{\mathbb{Q}}
 \newcommand{\RR}{\mathbb{R}}
 \newcommand{\ZZ}{\mathbb{Z}}
 \newcommand{\PP}{\mathbb{P}}

 %Derivatives
 \newcommand\der[2]{\frac{d #1}{d #2}}
 \newcommand\pder[2]{\frac{\partial #1}{\partial #2}}
 
 %Greek Letters
 \newcommand{\DDelta}{\Delta}
 \newcommand{\Ga}{\Gamma}
 \newcommand{\ga}{\gamma}
 \newcommand{\om}{\omega}
 \newcommand{\Om}{\Omega}
 
 %Matricies
 \newcommand{\brmat}[1]{\begin{bmatrix}#1\end{bmatrix}}
 
%%%%%%%%%%%% Begin Anton's Macros %%%%%%%%%%%%%%%%%%%%%%%%%%%%%%%%%%%%%

 % Small and Large Matricies
 \newcommand{\matx}[1]{\bigl(\begin{smallmatrix} #1 \end{smallmatrix}\bigr)}
 \newcommand{\Matx}[1]{\begin{pmatrix}#1\end{pmatrix}}
 
%\newcommand{\pb}{\rule{.4pt}{5.4pt}\rule[5pt]{5pt}{.4pt}\llap{$\cdot$\hspace{1pt}}}
%\newcommand{\po}{\rule{5pt}{.4pt}\rule{.4pt}{5.4pt}\llap{$\cdot$\hspace{1pt}}}
%\newcommand{\qbinom}[2]{\left[\frac{#1}{#2}\right]_q}

 \newcommand{\bbar}[1]{\setbox0=\hbox{$#1$}\dimen0=.2\ht0 \kern\dimen0 \overline{\kern-\dimen0 #1}}
 \renewcommand{\H}{\mathcal{H}} % old \H{x} is an x with a weird umlaut in text mode
 \newcommand{\hhat}[1]{\widehat{#1}}
 \renewcommand{\labelitemi}{--}  % changes the default bullet in itemize
 \newcommand{\inn}[1]{\accentset{\circ}{#1}}
 \newcommand{\lotimes}{\overset{L}{\otimes}}
 
%%%%%%%%%%%% End Anton's Macros %%%%%%%%%%%%%%%%%%%%%%%%%%%%%%%%%%%%%%%

 \openout0=lastupdated.html
 \write0{\today}
 \closeout0
 