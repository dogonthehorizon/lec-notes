
\sektion{3.2}{Class Lecture 10/3}
\subsektion{Homework Comments}
If we want to find bases for colspace($A$) and Null($A$)
\[ \Matx{1 & 0 & 1 & 0 & 5 & -5 \\ 2 & -2 & 0 & 5 & 6 & -1 \\ 1 & -5 & -4 & 0 & -3 & 3 \\ 5 & -4 & 1 & 2 & -1 & 3} \]
\[ \Matx{1 & 0 & 1 & 0 & 0 & 0 \\ 0 & 1 & 1 & 0 & 0 & 0 \\ 0 &0 &0 &1 &0 &1 \\ 0 &0 &0 &0 &1 &-1} \]
\begin{itemize}
\item Look at the pivots. Why is having a pivot important in every row? What does it say about the range? It spans all of $\RR^4$
\item Pivot in each row indicates that Col.Sp.($A$)=$\RR^4$. But more generally, the number of pivots rows indicates the rank($A$) of (dim(Col.Sp.($A$)))
\item If I wanted the basis for the column space, generically, we want the original columns
\item So if I look at the pivot columns, the first two, and then columns 4 and 5 are pivot colums. The basis for col.sp. comes from the original pivot columns.
\item If I want to look at the dimension of the column space, it is related to the number of free variables. Free variables correspond to non-pivot columns. Pertain to a basis for the Null($A$).
\item Basis for Null($A$): $x_3\Matx{-1 \\ -1 \\ 1 \\ 0 \\ 0 \\ 0}\ x_6\Matx{0 \\ 0 \\ 0 \\ -1 \\ 1 \\ 1}$
\item Basis: Linearly independent and it spans.
\end{itemize}

\subsektion{Cramer's Rules!}
If we have $A\vec{x}=\vec{b}$ satisfied $x_i=det(\vec{a_1}|\cdots|\vec{b}|\vec{a_n})$
\[ \text{To solve:} A \left( \vec{x_1}|\vec{x_2}|\cdots|\vec{x_n} \right) =I(\vec{e_1}|\vec{e_2}|\cdots|\vec{e_n}) \]

\[ (A^{-1})_{ij} = x_{ij}=\underbrace{det\left(\vec{a_1}|\cdots|\overbrace{\vec{e_j}}^{i\text{th}}|\cdots|\vec{a_n}\right)}_{det(A)} = \frac{1}{det(A)}\left( 1(-1)^{j+i}det(A_{ji})\right)\]

\begin{definition}
The adjoint of A is (adj($A$))$_{ij}$ = det($A_{ij}$)
\[ A^{-1}=\frac{1}{det(A)}\left(adj(A)\right)^T \]
\end{definition}
\begin{example}
Find $A^{-1}$ using this method.
\[ A = \brmat{3 & 5 & 1 \\  -1 & 1 & 4 \\ 2 & 1 & 0} \]
\[ det(A)=1(-1)^{1+3}\begin{vmatrix}-1 & 1 \\ 2 & 1 \end{vmatrix}
 + 4(-1)^{2+3}\begin{vmatrix}3 & 5 \\ 2 & 1\end{vmatrix}
  = -3 - 4(-7) = 25\]
\[ adj(A) = \Matx{-4 & 8 & -3 \\ 1 & -2 & 7 \\ 19 & -13 & 8} \]
\[ A^{-1} = \frac{1}{25}\Matx{-4 & 1 & 19 \\ 8 & -2 & -13 \\ -3 & 7 & 8}\]
\end{example}
\begin{definition}
\[ AA^{-1} =A^{-1}A = \frac{1}{det(A)}\left(adj(A)\right)^TA = I \] 
\end{definition}


