
\sektion{2}{Matrix Algebra}
\subsektion{Subspaces of $\RR$} %Section 2.8
\subsubsektion{Abstract}
The four fundamental concepts in this section are subspace, column space, null space, and basis. These techniques have been mechanically computed up until this point, however we will now understand the technical terminology behind them. These concepts will become essential in Chapters 5 \& 7.
\begin{definition}
A \textbf{subspace} of $\RR$ is any set $H$ in $\RR^n$ that has three properties:
	\begin{itemize}
		\item The zero vector is in $H$.
		\item For each $u$ and $v$ in $H$, the sume $u+v$ is in $H$.
		\item For each $u$ in $H$ and each scalar $c$, the vecture $cu$ is in $H$.
	\end{itemize}
\end{definition}
\begin{example}
If $v_1$ and $v_2$ are in $\RR^n$ and $H$ = Span$\{v_1,v_2\}$, then $H$ is a subspace of $\RR^n$. Note that the zero vector is in $H$ since $0v_1+0v_2$ is a linear combination of $v_1$ and $v_2$. Given two arbitrary vectors in $H$:
\[ u = s_1v_1 + s_2v_2\ \ \text{and}\ \ v = t_1v_1 + t_2v_2 \]
\[ \text{then}\ \ \ u+v = (s_1 + t_1)v_1 + (s_2+t_2)v_2 \]
this shows that $u+v$ is a linear combination of $v_1$ and $v_2$ and thus $H$. If $v_1$ is not zero and $v_2$ is a multiple of $v_1$, then $v_1$ and $v_2$ simply span a \emph{line} through the origin.
\end{example}
\subsubsektion{Column Space and Null Space of a Matrix}
\begin{definition}
The \textbf{column space} of a matrix $A$ is the set Col $A$ of all linear combinations of the columns of $A$. If $A=\left[ a_1, \cdots, a_n \right]$, with the columns of $\RR^m$, then Col $A$ is the same as Span$\{ a_1, \cdots, a_n\}$. $\therefore$ the column space of an $m\times n$ matrix is a subspace of $\RR^n$.
\end{definition}
\begin{example}
Let $A = \brmat{1 & -3 & -4 \\ -4 & 6 & -2 \\ -3 & 7 & 6}$ and $b = \brmat{3 \\ 3 \\ -4}$. $b$ is in the column space of $A$ $\iff$ $Ax = b$; that is, we need to row reduce $\brmat{A & b}$.
\[ \brmat{1 & -3 & -4 & 3 \\ -4 & 6 & -2 & 3 \\ -3 & 7 & 6 & -4} \sim \brmat{1 & -3 & -4 & 3 \\ 0 & -6 & -18 & 15 \\ 0 & -2 & -6 & 5} \sim \brmat{1 & -3 & -4 & 3 \\ 0 & -6 & 18 & 15 \\ 0 & 0 & 0 & 0} \]
and from this we determine that $Ax=b$ is consistent and $b$ is in Col $A$.
\end{example}
\begin{definition}
The \textbf{null space} of a matrix A is the set of Nul $A$ of all solutions of the homogeneous equation $Ax=0$.
\end{definition}
\begin{theorem}
The null space of an $m \times n$ matrix $A$ is a subspace of $\RR^n$. Equivalently, the set of all solutions of a system $Ax=0$ of $m$ homogenous linear equations in $n$ unknowns is a subpsace of $\RR^n$.
\end{theorem}
\subsubsektion{Basis for a Subspace}
\begin{definition}
A \textbf{basis} for a subspace of $H$ of $\RR^n$ is a linearly independent set in $H$ that spans $H$.
\end{definition}
\begin{example}Given $A = \brmat{-3 & 6 & -1 & 1 & -7 \\ 1 & -2 & 2 & 3 & -1 \\ 2 & -4 & 5 & 8 & -4}$ we want to find the basis of $A$. Begin by writing the solution of $Ax=0$ like so:
\[ \brmat{A & 0} \sim \brmat{1 & -2 & 0 & -1 & 3 & 0 \\ 0 & 0 & 1 & 2 & -2 & 0 \\ 0 & 0 & 0 & 0 & 0 & 0}, 
\begin{array}{rrrrrrrrrr} x_1 & - & 2x_2 &      & - & x_4  & + & 3x_5 & = & 0 \\ 
						      &   &      & x_3  & + & 2x_4 & - & 2x_5 & = & 0 \\
						      &   &      &      &   &      &   &    0 & = & 0 \end{array} \]
\[ \brmat{x_1 \\ x_2 \\ x_3 \\ x_4 \\ x_5} = 
   \brmat{2x_2 + x_4 - 3x_5 \\ x_2 \\ -2x_4 + 2x_5 \\ x_4 \\ x_5} =
   x_2 \underbrace{\brmat{2 \\ 1 \\ 0 \\ 0 \\ 0}}_u + x_4 \underbrace{\brmat{1 \\ 0 \\ -2 \\ 1 \\0}}_v + x_5 \underbrace{\brmat{-3 \\ 0 \\ 2 \\ 0 \\ 1}}_w = x_2u + x_4v + x_5w\]
The example above shows that Nul $A$ is generated by \{$u,v,w$\}, which automatically makes the set of linear equations linearly independent. According to theorem 12 all solutions of $Ax = 0$ are a subspace of $\RR^n$, and in order for this to be true $x_2,x_4,x_5$ must be zero when $0 = x_2u + x_4v + x_5w$. $\therefore$ \{$u,v,w$\} is a \emph{basis} for Nul $A$.
\end{example}
\begin{theorem}
The pivot columns of a matrix $A$ form a basis for the column space of $A$.
\end{theorem}

\subsektion{Dimension and Rank}
\subsubsektion{Abstract}
There are two fundamental concepts in this section, and unsurprisingly they are the dimension of a subspace and the rank of a matrix. The \textbf{Basis Theorem} and \textbf{Invertible Matrix Theorem} combines the concept of dimension. supbspace, linear independence, span, and basis.


