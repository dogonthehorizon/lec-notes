
\sektion{0}{Preface}
\subsektion{Interface Patterns}
Interfaces can be grouped into smaller pieces that are easily recognizable.
\begin{itemize}
\item Forms
\item Text Editors
\item Graphic Editors
\item Spreadsheets
\item Browsers
\item Calendars
\item Media Players
\item Information Graphics
\item Immersive Games
\item Web Pages
\item Social Spaces
\item E-Commerce Sites
\end{itemize}

\subsektion{Patterns in General}
Patterns are essentially structural and behavioral features that improve the \emph{habitability} of something; this could be anything from a website to a building. More specifically, patterns are:
\begin{itemize}
\item \emph{Concrete, not general}: Patterns should be concrete enough to fill the space between high-level general principles and low-level "grammar" of UX design (widgets, text, graphic elements, alignment grids, etc).

\item \emph{Valid across different platforms and systems}: Ideally, each pattern captures some minor truth about how people work best in a given situation and translate that truth across mediums.

\item \emph{Products, not process}: Patterns \emph{are} possible solutions, not ways to \emph{find} possible solutions (heuristics).

\item \emph{Suggestions, not requirements}: Patterns are intended to be suggestions, one should adjust as the contextual and user needs change.

\item \emph{Relationships among elements, not single elements}: A text field is a text field. A text field with help text is a pattern. Likewise changes in a set of elements over time may be a pattern.

\item \emph{Customized to each design context}: Fit the pattern to your particular users and requirements.
\end{itemize}

Complete sets of patterns make up a \emph{"pattern language"}. They're considered as such because they cover the complete set of elements and their relationsihps in a particular design.

\subsektion{Other Pattern Collections}
Pattern collections began with the "gold standard" titled \emph{A Pattern Language} by Christopher Alexander.

Beginning in the 1990s, several books on Software Engineering began to emerge, chief among them \emph{Design Patterns} by Erich Gamma, Richard Helm, Ralph Johnson, and John Vlissides.


