
\sektion{1}{Class Lecture - 10/09}
\subsektion{Test Debrief}
\subsubsektion{Chapter 1}
\begin{itemize}
\item (4pt) Computer Model
\item (4pt) Computer Generations
\item (3pt) Computer Categories
\item (5pt) Lifelong Learning
\item (4pt) System Performance: peak CPU, Benchmark
\end{itemize}

\subsubsektion{Chapter 2}
\begin{itemize}
\item CPU to Memory Interface
\item Address Bus
\item Data Bus
\item Control Bus: read/write
\item Memory Organizaiton
\item CPU Endian
\item CPU architectures: Von Neuman, Harard, Modified Harvard
\end{itemize}

\subsubsektion{J-Type}
\begin{itemize}
\item \textbf{j} jump ("one way ticket")
    \begin{itemize}
        \item 6 byte opcode followed by 26 byte absolute word address.
        \item syntax j 10000 \#goto memory address at 10000
    \end{itemize}
\item \textbf{jal} jump and link ("two way ticket")
    \begin{itemize}
        \item jal 2400 \#save return address in \$ra then go to memory address 2400
    \end{itemize}
\item \textbf{jr} jump register ("return ticket")
    \begin{itemize}
        \item jr \$ra \# return to the address stored in \$ra
    \end{itemize}
\end{itemize}

\subsubsektion{Stacks}
\begin{itemize}
\item Every program has a stack area for storing vital register values.
\item The stack is a LIFO queue pointed to by the \$sp.
\item The location pointed at by the \$sp is called the TOS (top of stack).
\item Stack grows downward.
\item You push onto the stack and pop off of the stack.
\end{itemize}



