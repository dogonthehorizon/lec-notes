\sektion{9}{Test Review 10/23}
10 votes, 5 candidates $\Matx{11 \\ 4}$

Fundamental Theorem of Arithmetic $\leftarrow$ strong induction

\begin{example}
If $n$ is an odd integer, $n^3$ is the sum of $n$ consecutive integers.
\end{example}

\begin{proof}
\begin{itemize}
\item Want to show that 27 is the sum of 3 consecutive integers. 8+9+7 = 27
\item Want to show that 125 is the sum of 5 consecutive integers. 27+26+25+24+23 = 125
\item $n^3$ = $n^2+1-n^2+2-\frac{n-1}{2}+n^2+n^2+1+n^2+2+\frac{n-1}{2}$
\item n-1 is even since n itself is odd, there we can divide by 2 and get an integer.
\item When we add them all up they turn into $n^3$.
\end{itemize}
\end{proof}

\begin{example}
Nobody doesn't like Sara Lee. What is the negation of this?
\[\exists x \in W (\not L(x) )\]
\[ \forall x \in W ( L(x) )\]
\end{example}

\begin{example}
\[ \forall n \in \NN \left[ P(n) \to Q(n) \right] \]
\begin{itemize}
\item Direct Proof:
    \begin{enumerate}
        \item Let $n\in\NN$. Assume $P(n)$.
        \item $\therefore$ $Q(n)$.
    \end{enumerate}
\item Contradiction Proof:
    \begin{enumerate}
    \item Let $n\in\NN$. Assume $P(n) \wedge \not Q(n)$.
    \item $\therefore$ $P(n)\to Q(n)$.
    \end{enumerate}
\item Contrapositive Proof:
    \begin{enumerate}
    \item Let $n\in\NN$. Assume $\not Q(n)$.
    \item $\therefore$ $\not P(n)$.
    \end{enumerate}
\end{itemize}
\end{example}

\begin{example}[5a]
\[ \forall x\in W \left( \exists r \in W \left( \exists p \in T \left( Q(x,r,p) \right) \right) \right) \]
\end{example}

\begin{example}[7]
Let $A,B$ be sets. Let them be subsets of the same universe $U$. Prove that if $A\subseteq B$, then $U \subseteq A \cup B$.
\end{example}

\begin{proof}
Let $A \subseteq U$ and $B \subseteq U$. Assume $A \subseteq B$. (To Show: $U \subseteq A \cup B$). Let $x\in U$.

By Cases:
\begin{itemize}
\item Case 1: $x\in \overline{A}$. $\therefore$ $x\in \overline{A}\cup B$.
\item Case 2: Since $x \in A \wedge \left( A \subseteq B \right)$ $x \in B$ $\therefore$ $x \in \overline{A} \cup B$.
\end{itemize}
\end{proof}

\begin{example}[6c]
\[ 12 | (n-k) \leftrightarrow \left( n = 12p + r \right) \wedge \left( k = 12q + r \right)\ 0\le r < 12 \]
\end{example}

\begin{proof}
Let $n,k\in\ZZ$. Assume $\left( n = 12p + r \right) \wedge \left( k = 12q + r \right)\ p,q\in\ZZ$. (To Show: $12|(n-k)$)
\begin{itemize}
\item $(n-k) = \left( 12p + r \right) - \left( 12q + r \right)$
\item $12p-12q$
\item $12(p-q)$
\item $\therefore$ by definition $12|(n-k)$.
\end{itemize}
Assume that the remainders are not equal. (To Show: $12 \not |\ (n-k) $)
\begin{itemize}
\item $n=12p+r_1 \wedge k=12q+r_2 \wedge 0 \le r_1,r_2 < 12 \wedge r_1 \neq r_2$.
\item $(n-k)=(12(p-q)+(r_1-r_2)$
\item If $r_1 > r_2$, then $0 < r_1-r_2 < 12$.
\item $\therefore$ since $r_1 \neq r_2$ and $(n-k)=(12(p-q)+(r_1-r_2)$ then $12 \not |\ (n-k) $.
\end{itemize}
\end{proof}

\textbf{To Know:}
\begin{itemize}
\item set theory proof or stuff
\item number theory proof or stuff
\item counting problem
\item formal logic stuff
\end{itemize}



