\sektion{2}{Class Lecture 9/25}

Def$^n$: Prime \\
Def$^n$: Relatively Prime \\

\textbf{Axiom:} For any non-zero ratio r=$\frac{p}{z}$, where $p,z \in \ZZ$, p and q are relatively prime.


\textbf{Thm. 2.2.2:} $\sqrt{2}$ is irrational.
\\ \\
\textbf{Proof:} 
\begin{enumerate}
\item \underline{By contradiction.} Assume $\sqrt{2}$ is rational. 
\item $\therefore \ \sqrt{2} = \frac{p}{z}$, where p and q are relatively prime.
\item $\left( \sqrt{2} \right)^2 = (p)^2$
\item $\left( \sqrt{2} \right)^2 (q)^2 = (p)^2$
\item $2q^2 = p^2$
\item $\therefore\ p^2$ is even. What does tat about p?
\item Because odd odd = odd, p must be even
\item $\therefore\ 2|p$, so $p=2k$ for some $k\in U$
\item $2q^2 = (2k)^2 \to 2q^2 = 4k^2 \therefore q^2 = 2k^2$
\item By the same arguments as above, q is even $\therefore\ 2|q$. So $2|p$ and $2|q$.
\item Hence, p and q are not relatively prime. So p and q are relatively prime and p and q are not relatively prime.
\item $\therefore$ Our assumption that $\sqrt{2}$ is ratiional must be false; i.e. $\sqrt{2}$ is irrational. \textbf{QED}
\end{enumerate}

\textbf{Thm:} There are irrational  numbers $r,s$ so that $r^s$ is rational.
\\
\\
\textbf{Proof}
\begin{enumerate}
\item Consider $r=\sqrt{2}$ and $s=\sqrt{2}$. Look at $r^s=\sqrt{2}^{\sqrt{2}}$
\begin{enumerate}
	\item $\sqrt{2}^{\sqrt{2}}$ is rational. Done this case! $r=p$ and $s=q$
	\item $\sqrt{2}^{\sqrt{2}}$ is irrational: Consider $r=\sqrt{2}^{\sqrt{2}}$ and $s=\sqrt{2}$. Double rational needs fixing%$r^2=\left\sqrt{2}^{\sqrt{2}}\right^{\sqrt{2}}$. \textbf{QED}
	%troubleshoot double superscript
\end{enumerate}
\end{enumerate}

\subsektion{More Skipping Around}
\textbf{Axiom:} Well-Ordering Princple -- Every non-empty subset of $\mathbb{N}$ contains a least element
\\
\[ \forall A \subseteq \mathbb{N} \left[\ A \neq \emptyset \to \exists x \in A \left( \forall X \in A \left( x <= y \right) \right)\ \right]\]

Thm 2.2.4 Every natural number $>$ 1 is divisible by some prime. 
\[\forall n \in \mathbb{N} \left( n > 1 \to \exists p \in \mathbb{N} \left( p is prime and p|n \right) \right) \] %Fix the last statement
\\ \\
Proof: 
\begin{enumerate}
\item Let $n\in\mathbb{N}$. Assume $n>1$.
\item Case 1: $n$ is prime. 
\begin{enumerate}\item $n|n$, because $n = n \times 1$. Done \end{enumerate}
\item Case 2: $n$ is composite.
\begin{enumerate}
	\item $\therefore$ $n$ has a non-trivial divisor
	\item $\therefore$ there is a d such that $1 < d < n$ and $d|n$
	\item Let $A = \{ x | 1 < x < n\ and\ x|n \}$
	\item So $d \in A\ \therefore A \neq \emptyset$
\end{enumerate}
\end{enumerate}

\subsektion{Quiz 2 Review}
3b. $\exists x \in D \left( \forall y \in U  \left( f(\pi y) = 0 \right) \right)$ 

%diagram of a triangle, sqrt(2) hypotenus, 1 on legs


