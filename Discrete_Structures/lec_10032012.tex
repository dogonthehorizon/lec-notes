\sektion{4}{Class Lecture 10/3}
\subsektion{Strong Induction Cont.}
\begin{proof}[Every $n \ge 2$ has a prime divisor.]
By Strong Induction.
\begin{itemize}
\item Base Case: $n=2$. 2 is prime and $2|2$. \ck
\item Induction Step: Let $k \ge 2$. Assume $2,\cdots,k$ all have prime divisors.
    \begin{enumerate}
        \item To Show: $k+1$ has a prime divisor.
        \item Case 1: $k+1$ is prime. $(k+1)|(k+1)$ \ck
        \item Case 2: $k+1$ is composite. 
        \begin{itemize}
            \item $\therefore$ there is some $m\in \NN$ such that $1<m<k+1$ and $m|(k+1)$
            \item $2 \le m \le k$
            \item By Induction Hypothesis, $m$ has a prime divisor $p$.
            \item $p|m \wedge m|(k+1)$
            \item By the transitivty of divisibility $p|(k+1) \wedge p$ is prime. \ck
        \end{itemize}
    \end{enumerate}
\end{itemize}
\end{proof}

\subsektion{Greatest Common Divisor}
\begin{example}[Find the greatest common divisor of 20 and 36]
\[ \begin{matrix}20 & = & 2 & \times & 2 & \times & 5 \\
         36 & = & 2 & \times & 3 & \times & 3 & \times & 3\end{matrix}\]
\[\underbrace{gcd(20,36)=4}_{\text{Terrible Algorithm. Fact.}}\]
\end{example}
\begin{definition}
Let $n,m\in \ZZ; n,m \ge 1$. The gcd($n,m$)=$d \iff (d|m \wedge d|n) \wedge \left( \forall c \in \NN \left( (c|m \wedge c|n) \to c \le d \right) \right)$

\begin{itemize}
\item $n=mq+r$, $0\le r \le m$
\item What if $a|n$ and $a|m$? $a|(n-mq)$, so $a|r$.
\item What if $a|m$ and $a|r_1$? $a|n$
\item $n=mq_1+r_1$
\item $m=r_1q_2+r_2$
\item $r_1=r_2q_3+r_3$
\item $m > r_1 > r_2 > \cdots = 0$
\item gcd($n,m$) = gcd($m,r_1$) = gcd($r_1,r_2$) = $cdots$ = last non-zero remainder.
\end{itemize}
\end{definition}
\begin{example}[ Find $gcd(12600,6825)$ ]
\textbf{Euclidian Algorithm}
\[ \begin{matrix} 12600 & = & 6825(1) & + & 5775 \\
                  6825  & = & 5775(1) & + & 1050 \\
                  5775  & = & 1050(5) & + & 525  \\
                  1050  & = & 525(2)  & + & 0\end{matrix} \]
\[ gcd(12600,6825)=525 \]
\end{example}

\begin{definition}
Let $n\in\ZZ$. An (integer) linear combination of $n,m$ is a number/expression of the form $nx+my$, where $x,y\in\ZZ$.
\end{definition}

\begin{theorem}
Let $n,m\in\ZZ; n,m\ge 1$. Then gcd($n,m$) is the smallest positive number that can be written as a linear combination of $n,m$.
\end{theorem}





