\sektion{16}{Class Lecture 11/16}
\noindent
Definition: A \emph{bijection} $F:A\to A$ is called a permutation of $A$ \\
FACT: If $A,B$ are finite with $|A|=|B|$ and $F:A\to B$ then $F$ is 1-1 iff $F$ is onto.
\begin{example}
The above fact is not necessarily true if $A,B$ are infinite. For instance, let $A=B=\NN$. Let $F:\NN\to\NN$ be given by $F(n)=n^2$. Then $F(0)=0$, $F(1)=1$, $F(2)=4$, $F(3)=9$. Is $F$ 1-1? Yes. Is $F$ onto? No. 
\end{example}
\noindent
Definition: Let $A,B$ be any sets $|A|=|B|$ if and only if there is some bijection $F:A\to B$. \\
Definition: An infinite set of $A$ is countable infinite if $|A|=|\NN|$.
\begin{example}
$\ZZ$ is countably infinite
\[ \begin{matrix}
        \cdots & -3 & -2 & -1 & 0 & 1 & 2 & 3 & \cdots & \ZZ \\
               &  6 &  4 &  2 & 0 & 1 & 3 & 5 &        & \NN
   \end{matrix}
\]
\end{example}
\begin{theorem}[Schroeder-Bernstein Theorem]
If $A,B$ are sets and $F:A\to B$ is 1-1 and $G:B\to A$ is 1-1 then there is a bijection $H:A \to B$.
\end{theorem}
\begin{example}
$\QQ^{\ge 0}$ is countably infinite. $F:\NN\to\QQ^{\ge 0}$ is given by $F(n)=n$ and is 1-1. Let $G:\QQ^{\ge 0}\to\NN$ be given as follows: $G(0)=0$, $G(\frac{n}{m})=2^n3^m\in\NN$. \\
Is $G$ 1-1? Assume $G(\frac{n_1}{m_1})=G(\frac{n_2}{m_2})$. $\therefore$ $2^{n_1}3^{m_1}=2^{n_2}3^{m_2}$. By the Fundamental Theorem of Arithmetic $n_1=n_2$ and $m_1=m_2$. $\therefore$ $\frac{n_1}{m_1}=\frac{n_2}{m_2}$. \ck $\therefore$ by the Schroder-Bernstein Theorem $\QQ^{\ge 0} \to \NN$.
\end{example}


