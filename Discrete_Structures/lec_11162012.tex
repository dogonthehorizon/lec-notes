\sektion{18}{Class Lecture 11/16}
\begin{proposition}
Let $F:A\to B$, $G:B\to C$ be functions. If $G\cdot F:A \to C$ is onto, then $G$ must be onto.
\end{proposition}
\begin{proof}
Let $F:\RR\to\RR$ be given by $F(r)=r^2$, $G:\RR\to\RR$ given by 
\[ G(r)=\left\{\begin{matrix} 0 & r = 0 \\ \ell n(r^2) & r \neq 0 \end{matrix} \right. \]
\[ G\cdot F(r^2)=\left\{\begin{matrix} 0 & r = 0 \\ \ell n(r^2) & r \neq 0 \end{matrix} \right. \]
\end{proof}
\subsektion{Finite Permutations}
\begin{itemize}
\item Let $A_n=\left\{1,2,\cdots,n\right\}$. $S_n=\left\{\pi : A_n \to A_n\ |\ \pi \text{ is a permutation; i.e.} \pi \text{ is 1-1 and onto}\right\}$
\item \emph{Disjoint Cycle Notation:} $\pi$ = (1 3 2) (4 5)\footnote{1 is mapped to 3, 3 is mapped to 2, two is mapped to 1 so we close parens. 4 is mapped to 5, 5 is mapped to 4 so we close the parens.}
\item $\rho,\pi\in S_n$, $\rho\pi=\rho\cdot\pi$
\end{itemize}
\begin{example}
\[ \begin{matrix} \rho: &1 &2 &3 &4 &5 \\
                        &\downarrow&\downarrow&\downarrow&\downarrow&\downarrow \\
                        &5     &4     &2     &1     &3
   \end{matrix}
\]
\[ \pi: (1\ 5\ 3\ 2\ 4) \]
\end{example}
\begin{example}
\[ \begin{matrix} \rho\pi= &1 &2 &3 &4 &5 \\
                        &\downarrow&\downarrow&\downarrow&\downarrow&\downarrow \\
                        &2     &5     &4     &3     &1
   \end{matrix}
\]
\end{example}
\begin{example}
\[ \rho\pi = (1\ 5\ 3\ 2\ 4\ 1)(1\ 3\ 2) (4\ 5)\]
\[ (1\ 2\ 5)(3\ 4)
\]\footnote{This is equivalent to the result from the previous example.}
\end{example}
\subsektion{Pigeon-Hole Principle}
\noindent
Fact: If $A,B$ are finite sets, $|A|>|B|$ and $F:A\to B$, then $F$ is not 1-1; i.e. there are $a_1,a_2\in A$ such that $a_1 \neq a_2$ but $F(a_1)=F(a_2)$.\footnote{If you have $n$ pidgeons and $m$ holes, then there is a case where two pidgeons fly into the same hole, thus the name.}
\begin{example}
Assume that being friends is symmetric. Then there are at least two people in MTH311A who have the same number of friends in MTH311A.
\end{example}
\begin{proof}
Let $A$ = set of all MTH311A students. $|A|$=26. $B=\left\{0,\cdots,25\right\}$. $|B|=26$. $F:A\to B$ such that $f(a)$= \# of friends in MTH311A. Since being a friend is symmetric, it can't be the case that $F(a_i)=0$ and $F(a_j)=25$ for some $i,j$.
\end{proof}
\noindent
\textbf{\emph{Generalized P-H-P:}} Let $A,B$ be sets $F:A\to B$, $k\in\NN$,$k\ge1$. If $|A|>k\cdot|B|$, then there must be $k+1$ distinct elements of $A$, $a_1,a_2,\cdots,a_{k+1}$ so that $F(a_1)=F(a_2)=\cdots=F(a_{k+1})$.


