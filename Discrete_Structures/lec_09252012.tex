\sektion{1}{Class Lecture 9/25}

\begin{definition}
For any non-zero ratio r=$\frac{p}{z}$, where $p,z \in \ZZ$, p and q are relatively prime.
\end{definition}
\begin{theorem}[Textbook 2.2.2]
$\sqrt{2}$ is irrational.
\end{theorem}
\begin{proof}\underline{By contradiction.} Assume $\sqrt{2}$ is rational. $\therefore \ \sqrt{2} = \frac{p}{z}$, where $p$ and $q$ are relatively prime. $\left( \sqrt{2} \right)^2 = (p)^2$. $\left( \sqrt{2} \right)^2 (q)^2 = (p)^2$. $2q^2 = p^2$. $\therefore\ p^2$ is even. What does that say about p?. Because odd$\cdot$odd = odd, p must be even. $\therefore\ 2|p$, so $p=2k$ for some $k\in U$. $2q^2 = (2k)^2 \to 2q^2 = 4k^2 \therefore q^2 = 2k^2$. By the same arguments as above, $q$ is even $\therefore\ 2|q$. So $2|p$ and $2|q$. Hence, p and q are not relatively prime. So p and q are relatively prime and p and q are not relatively prime. $\therefore$ Our assumption that $\sqrt{2}$ is rational must be false; i.e. $\sqrt{2}$ is irrational.
\end{proof} 
\begin{theorem}
There are irrational  numbers $r,s$ so that $r^s$ is rational.
\end{theorem}
\begin{proof} By Cases.
\begin{enumerate}
\item Consider $r=\sqrt{2}$ and $s=\sqrt{2}$. Look at $r^s=\sqrt{2}^{\sqrt{2}}$
\begin{enumerate}
	\item $\sqrt{2}^{\sqrt{2}}$ is rational. \ck $r=p$ and $s=q$
	\item $\sqrt{2}^{\sqrt{2}}$ is irrational: Consider $r=\sqrt{2}^{\sqrt{2}}$ and $s=\sqrt{2}$. $r^2=\left( \sqrt{2}^{\sqrt{2}}\right)^{\sqrt{2}}$. \ck
\end{enumerate}
\end{enumerate}
\end{proof}
\subsektion{Well Ordering Principle}
\begin{theorem}[Well Ordering Principle]
Every non-empty subset of $\NN$ contains a least one element.

\[ \forall A \subseteq \NN \left[\ A \neq \emptyset \to \left( \exists x \in A \left( \forall x \in A \left( x <= y \right) \right)\right)\ \right]\]
\end{theorem}
\begin{theorem}[Textbook 2.2.4]
Every natural number $>$ 1 is divisible by some prime. 
\[\forall n \in \mathbb{N} \left( n > 1 \to \exists p \in \mathbb{N} \left( p \text{ is prime and } p|n \right) \right) \]
\end{theorem}
\begin{proof}
\begin{enumerate}
\item Let $n\in\NN$. Assume $n>1$.
\item Case 1: $n$ is prime. 
\begin{enumerate}\item $n|n$, because $n = n \cdot 1$. \ck \end{enumerate}
\item Case 2: $n$ is composite.
\begin{enumerate}
	\item $\therefore$ $n$ has a non-trivial divisor
	\item $\therefore$ there is a d such that $1 < d < n$ and $d|n$
	\item Let $A = \{ x | 1 < x < n\ and\ x|n \}$
	\item So $d \in A\ \therefore A \neq \emptyset$
\end{enumerate}
\end{enumerate}
\end{proof}


